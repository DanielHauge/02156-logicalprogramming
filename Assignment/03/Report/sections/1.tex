\newpage
\section*{Problem 1}

\subsection*{Question 1.1}
The following query:

\begin{verbatim}
    ?- tt(p imp q, [(p,t), (q,f)], f).
    True .
\end{verbatim}

Is showing that, given $p$ is true and $q$ is false then ($p \to q = f$) is true. 

\subsection*{Question 1.2}
The boolean/0 program only succeeds if all possible sematics are of two-valued boolean are satisfied.
Using the predicate tt in the same vein as in Question 1.1, two-valued boolean logic can be checked. The following query checks that negation works as required:
\begin{verbatim}
    ?- tt(neg p, [(p, t)], f), tt(neg p, [(p,f)], t).
\end{verbatim}

\subsection*{Question 1.3}
The many valued logic is based on the semantics defined in assignment 1. The semantics are as follows:

$$ [[\neg P]] = 
    \begin{cases}
        \textrm{T} & \textrm{if } [[P]] = \textrm{F} \\
        \textrm{F} & \textrm{if } [[P]] = \textrm{T} \\
        [[P]] & \textrm{otherwise}    
    \end{cases}
$$

$$ [[P \land Q]] = 
    \begin{cases}
        [[P]] & \textrm{if } [[P]] = [[Q]] \\
        [[Q]] & \textrm{if } [[P]] = \textrm{T} \\
        [[P]] & \textrm{if } [[Q]] = \textrm{T} \\
        \textrm{F} & \textrm{otherwise} 
    \end{cases}
$$

$$ [[P \leftrightarrow Q]] = 
    \begin{cases}
        \textrm{T} & \textrm{if } [[P]] = [[Q]] \\
        [[Q]] & \textrm{if } [[P]] = \textrm{T} \\
        [[P]] & \textrm{if } [[Q]] = \textrm{T} \\
        [[\neg Q]] & \textrm{if } [[P]] = \textrm{F} \\
        [[\neg P]] & \textrm{if } [[Q]] = \textrm{F} \\
        F & \textrm{otherwise} 
    \end{cases}
$$

$$[[P \lor Q]] \equiv \neg (\neg P \land \neg Q)$$
$$[[P \to Q]] \equiv P \leftrightarrow (P \land Q) $$

\subsubsection*{Testing}
For more extensive testing, see appendix.
\code{?- opr(imp, t, x, x). \\
    true.\\
    ?- opr(con, x, x, x). \\
    true.\\
    ?- opr(dis, f, x, x).\\
    true.\\
    ?- opr(eqv, x, x, t).\\
    true.
}

\subsection*{Question 1.4}
The semantics of many valued logic still contain the requirements for satisfying two-valued logic, thus boolean/0 still succeed.