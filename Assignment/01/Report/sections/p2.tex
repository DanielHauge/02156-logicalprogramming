\newpage
\section*{Problem 2}

\subsection*{Question 2.1}
Refuting the validity of the proposition, we negate the formular and try to find counter example:

\Tree[.\textit{$\neg (((s \to p) \land ((\neg s \to t) \land (\neg p \to \neg t))) \to p)$} 
        [ .\textit{$((s \to p) \land ((\neg s \to t) \land (\neg p \to \neg t))) , \neg p$} 
            [.\textit{$s \to p , (\neg s \to t) \land (\neg p \to \neg t), \neg p$}
                [.\textit{$p , (\neg s \to t) \land (\neg p \to \neg t), \neg p$} \textit{$\times$} ] 
                [.\textit{$\neg s , (\neg s \to t) \land (\neg p \to \neg t), \neg p$} 
                    [.\textit{$\neg s , (\neg s \to t) , (\neg p \to \neg t), \neg p$} 
                        [.\textit{$\neg s , s, (\neg p \to \neg t), \neg p$} \textit{$\times$} ]
                        [.\textit{$\neg s , t, (\neg p \to \neg t), \neg p$} 
                            [.\textit{$\neg s , t, p, \neg p$} \textit{$\times$} ] 
                            [.\textit{$\neg s , t, \neg t, \neg p$} \textit{$\times$} ] 
                        ]
                    ]  
                ] 
            ]
        ]
]

Concluding with a complete closed tableau, thus we can say that the formula is valid hence we can say the propositional formular is a tautology.

\subsection*{Question 2.2}
Considering the logical equivalence, we observe that it can be used to argument for the following:
$$ s \to p \equiv \neg p \to \neg s$$
and
$$ \neg p \to \neg t \equiv  t \to p$$
Therefor we can "swap" the parts marked with underline below to show equivalence with the formular:
$$ ((\underline{s \to p}) \land ((\neg s \to t) \land (\underline{\neg p \to \neg t}))) \to p $$
