\newpage
\section{Problem 3}

\subsection{Question 3.1}

See the prolog file for the predicate.

The predicate should fail when Elem is not occuring twice.
\code{?- member2(1,[1,2,3,4]).\newline false.}

The predicate should succed when 2 occurences of Elem are present in List.
\code{?- member2(1,[1,2,3,4,1]).\newline true ;\newline false.}

The predicate can be used as quiery to find elements occuring twice in List.
\code{?- member2(X,[1,2,3,4,2,4]).\newline X = 2 ;\newline X = 4 ;\newline false.}

The predicate will give multiple answers on backtracking.
\code{?- member2(3, [3,3,3]).\newline true ;\newline true ;\newline true ;\newline false.}

The predicate can give potentially infinite answers for lists which has Elem twice occuring.
\code{?- member2(1,X).\newline X = [1, 1|\_358] ;\newline X = [1, \_1022, 1|\_1036] .}

The predicate can used in a query with variables only. Although is seems pointless to do so.
\code{?- member2(X,A).\newline A = [X, X|\_2830] ;\newline A = [X, \_3564, X|\_3578] .}


\subsection{Question 3.2}
See the prolog file for the predicate. All tests from Question 3.1 has been done with $member2a$. All tests give identical results (apart from generic values).