\section*{Problem 1}


\subsection*{Question 1.1}

Truth table is constructed using semantics written in the assignment description.

\[
\begin{array}{C|C|C|C|C|C|C}
$A$ & $B$ & $\neg A$ & $A \land B$ & $A \lor B$ & $A \to B$ & $A \leftrightarrow B$\\
\hline
T & T & F & T & T & T & T \\
T & F & F & F & T & F & F \\
T & X & F & X & T & X & X \\
F & T & T & F & T & T & F \\
F & F & T & F & F & T & T \\
F & X & T & F & X & T & $\neg$ X \\
X & T & $\neg$ X & X & T & T & X \\
X & F & $\neg$ X & F & X & $\neg$ X & $\neg$ X\\
X & X & $\neg$ X & X & X & T & T 
\end{array}
\]

The lack of semantics for $X$ will make some cases just terminate with $\neg X$. 
An interesting observation from comparison is that some logical operations disregard or will work just fine without classical truth values. 
Like implication $X \to X$ will give True, as with implication it does not matter what value it is operating with.  


\subsection*{Question 1.2}
When p is T:

$$ \neg T \land T = F \land T = \doubleunderline{F} $$

When p is F:

$$ \neg F \land F = T \land F = \doubleunderline{F} $$

When p is X:

$$ \neg X \land X = \doubleunderline{F} $$

As there is no semantics for the negation of X, there is no better evaluation of ¬X. Hence we conclude the X case to be false, as the values is not equal and neither of them is T.

